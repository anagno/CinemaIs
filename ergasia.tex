%xelatex -shell-escape -output-directory=bin ergasia.tex
\documentclass{assignment}

\usepackage{enumerate} % Για την χρησιμοποίηση roman enumerate
\usepackage{paralist} % για το περιβάλλον inparaenum που είναι οι λίστες μέσα στο κείμενο.

\title{Πληροφοριακό σύστημα για online κρατήσεις θέσεων της εταιρείας ΚΙΝΗΜΑΤΟΓΡΑΦΟΣ Α.Ε.}
\date{Αθήνα, 2014}

\author{Αναγνωστόπουλος Βασίλης - Θάνος (ΜΠΠΛ 13002) \\ Κατσής Γιώργος (ΜΠΠΛ 13035)}

\begin{document}

\maketitle
% Να σκεφτώ τί αλλαγές θέλω να κάνω με τις αριθμήσεις και άμα θέλω να κάνω.
% Να σκεφτώ να τις ενσωματώσω και στο assignment.cls

\setcounter{page}{1} 
\pagenumbering{roman}

\pagestyle{plain}
\tableofcontents
%\listoftables
\listoffigures
%\renewcommand\listoflistingscaption{Κατάλογος πηγαίου κώδικα}
%\listoflistings
\newpage

%\pagestyle{headings}
%\pagestyle{fancy}
\setcounter{page}{1} 
\pagenumbering{arabic}

\section{Εισαγωγή}

Η εργασία αυτή έχει ως σκοπό την σχεδίαση και ανάλυση ενός πληροφοριακός συστήματος. Συγκεκριμένα η παρούσα αναφορά περιγράφει την βασική λειτουργικότητα και τις σχεδιαστικές αποφάσεις που αφορούν την υλοποίηση του Πληροφοριακού Συστήματος (Π.Σ.) της εταιρείας "ΚΙΝΗΜΑΤΟΓΡΑΦΟΣ Α.Ε." . Θα παρουσιαστούν οι λειτουργικές και οι μη λειτουργικές απαιτήσεις του συστήματος, το περιβάλλον το οποίο θα χρησιμοποιείται καθώς και οι χρήστες του. Στην συνέχεια θα γίνει μοντελοποίηση του συστήματος με την βοήθεια διαγραμμάτων οντοτήτων συσχετίσεων, διαγραμμάτων κλάσεων, διαγραμμάτων ροής και περιπτώσεων χρήσης.

\subsection{Περιβάλλον του έργου και σκοπός δημιουργίας του πληροφοριακού συστήματος}

%Θα πρέπει να προσδιοριστούν οι στόχοι που θα υλοποιηθούν από το σύστημα, οι προδιαγραφές που θα έχει και οι περιορισμοί στους οποίους θα πρέπει να συμμορφώνεται. Θα πρέπει επίσης να διασφαλιστεί ότι θα ικανοποιηθούν οι ανάγκες των ενδιαφερόμενων και για να γίνει αυτό θα πρέπει να οριστούν με ακρίβεια οι λειτουργικές και οι μη λειτουργικές απαιτήσεις.

% Το περιβάλλον περιγράφει την εταιρεία και ο σκοπός είναι για ποιον λόγο δημιουργείται.

Η εταιρεία "ΚΙΝΗΜΑΤΟΓΡΑΦΟΣ Α.Ε.", που από εδώ και πέρα απλώς θα καλείται "εταιρεία", δραστηριοποιείται στην διανομή κινηματογραφικών έργων στους θερινούς κινηματογράφους της Αττικής και της Θεσσαλονίκης. Η εταιρεία αναπτύσσει την επιχειρηματική της δραστηριότητα σύμφωνα με την ελληνική νομοθεσία που διέπει της ανώνυμες εταιρείες.

Βασικός σκοπός της εταιρείας είναι:

\begin{itemize}
\item Η προώθηση των παλιών και νέων κινηματογραφικών έργων στους θερινούς κινηματογράφους. 
\item Η είσπραξη των πνευματικών δικαιωμάτων από τις κινηματογραφικές προβολές και η απόδοση τους στις εταιρείες διανομής.
\item Η οργάνωση των κυλικείων των θερινών κινηματογράφων για να είναι κερδοφόροι.
\item ΝΑ ΣΚΕΦΤΟΥΜΕ ΚΑΙ ΤΠΤ ΑΛΛΟ !!!
\end{itemize}

Με την διείσδυση των νέων τεχνολογιών στην καθημερινότητα, οι άνθρωποι χρησιμοποιούν όλο και περισσότερο το διαδίκτυο για την πραγματοποίηση απλών καθημερινών διαδικασιών των ανθρώπων. Παρά το γεγονός ότι η χρήση του internet παραμένει χαμηλή στη Ελλάδα συγκριτικά με την Ευρώπη, σχεδόν ένας στους πέντε Έλληνες (ποσοστό 20,08\%) χρησιμοποιεί πια το διαδίκτυο, ενώ το 17,9\% του πληθυσμού το χρησιμοποιεί τακτικά τουλάχιστον μια φορά την εβδομάδα. Οι νεαρότερες ηλικιακά ομάδες (16-24 ετών: 42\%, 25-34 ετών: 30\%) και οι κάτοικοι των αστικών πόλεων με ανώτερη μόρφωση, αποτελούν με σημαντική διαφορά τις ομάδες πληθυσμού με την υψηλότερη πρόσβαση \cite{infosoc}. 

Βασισμένοι στα παραπάνω, η εταιρεία θεωρεί σημαντικό για την προώθηση των θερινών κινηματογράφων να αναπτυχθεί ένα σύστημα λογισμικού για την κράτηση θέσεων μέσω διαδικτύου.

\subsection{Σκοπός και στόχος του Π.Σ.}

Σκοπός της συγκεκριμένης μελέτης είναι η υλοποίηση ενός Π.Σ. που αποσκοπεί στην προώθηση των θερινών κινηματογράφων της εταιρείας και των ταινιών που προβάλλουν. Ακόμα μέσω του Π.Σ. θα είναι δυνατή η κράτηση θέσεων μέσω του διαδικτύου και η αποπληρωμή των εισιτηρίων με ηλεκτρονικούς τρόπους πληρωμής (π.χ. πιστωτικές κάρτες, paypal, κ.λ.π.). Το Π.Σ. αναμένεται να αποφέρει οφέλη στους τομείς:

\begin{itemize}
\item της διαφήμισης, μίας και η ιστοσελίδα για την κράτηση των θέσεων θα αποτελεί και ταυτόχρονα μέσω προώθησης της εταιρείας
\item της εξυπηρέτησης των πελατών, μίας και οι πελάτες δεν θα πρέπει να περιμένουν στην σειρά για την απόκτηση θέσης
\item της οργάνωσης της εταιρείας, μίας και θα δημιουργηθεί ένα αυτόματο σύστημα επεξεργασίας της διαθεσιμότητας των θέσεων
\item ΝΑ ΣΚΕΦΤΟΥΜΕ ΚΑΙ ΤΠΤ ΑΛΛΟ !!!
\end{itemize}

Η επιτυχία του έργου θα κριθεί κυρίως από το εύρος χρήσης του και από την αξιοποίηση των εξειδικευμένων δυνατοτήτων του, που αποσκοπούν κύρια στην αυτοματοποίηση του συστήματος για την κράτηση θέσεων.

Μετά την ολοκλήρωση του έργου του Π.Σ. θα ωφεληθούν άμεσα:

\begin{itemize}
\item οι εργαζόμενοι της εταιρείας μίας και θα απλοποιηθεί η διαδικασία για την κράτηση θέσεων
\item και οι θερινοί κινηματογράφοι της εταιρείας μίας και θα έχουν έναν εύκολο τρόπο να διαφημίζουν τις προβολές τους.
\end{itemize}

\subsection{Υφιστάμενη κατάσταση}

Η εταιρεία απαρτίζεται από τον πρόεδρο, τον αντιπρόεδρο και άλλους 3 εργαζόμενους πλήρης απασχόλησης. Αν και δεν υπάρχει καταρτισμένο οργανόγραμμα της εταιρείας, οι δύο από τους τρεις εργαζόμενους πλήρης απασχόλησης ασχολούνται με τα λογιστικά της εταιρείας και ο τελευταίος, μαζί με τον πρόεδρο και τον αντιπρόεδρο της εταιρείας ασχολούνται με την προώθηση των έργων στους θερινούς κινηματογράφους και την εξυπηρέτηση των πελατών.

Η εταιρεία διαθέτει ήδη ένα Πληροφορικό Σύστημα για την Διαχείριση (αγγλ. MIS) του λογιστηρίου και η διασύνδεση του με το νέο Π.Σ. για τις online κράτησεις θέσεων θα ήταν θετικό βήμα αλλά δεν κρίνεται απαραίτητο από τους εργαζόμενους.

\subsection{Βασικές οντότητες και εμπλεκόμενοι στην υλοποίηση του έργου}

Οι βασικοί εμπλεκόμενοι στην υλοποίηση του έργου είναι οι πελάτες του κινηματογράφου αλλά και οι υπάλληλοι των κινηματογράφων. Το άμεσο περιβάλλον του έργου παριστάνεται στο σχήμα \ref{fig:entities}. Οι οντότητες αυτές αναλύονται παρακάτω. 

\begin{figure}
\begin{center}
\resizebox*{\textwidth}{!}{
\includegraphics{images/entities.png}}
\caption{Οι βασικές οντότητες του Πληροφορικού Συστήματος}
\label{fig:entities}
\end{center}
\end{figure}

Ως πελάτες ορίζονται όλοι όσοι επιθυμούν να δουν κάποια ταινία στους κινηματογράφους, ενώ ως κινηματογράφος ορίζεται το κτήριο στο οποίο υπάρχουν οι αίθουσες προβολής. Επομένως ένα κτήριο μπορεί να περιλαμβάνει παραπάνω από μία αίθουσες προβολής, αλλά παρόλα αυτά θα θεωρείται ως ένας κινηματογράφος. Τέλος ως οντότητα "κράτηση θέσεων" θεωρούμε το πληροφοριακό σύστημα στο οποίο θα γίνονται οι κρατήσεις των θέσεων.

Άρα η βασική απαίτηση του συστήματος (δηλαδή η επιχειρηματική απαίτηση του Π.Σ.) είναι η γεφύρωση του χάσματος μεταξύ των πελατών και των κινηματογράφων. Φέρνοντας σε επικοινωνία τις δύο οντότητες η κράτηση των θέσεων απλοποιείται και τα έσοδα της εταιρείας αυξάνονται.

Στις παρακάτω ενότητες θα περιγραφούν οι προδιαγραφές και οι περιορισμοί στους οποίους θα πρέπει να συμμορφώνεται το υπό μελέτη πληροφοριακό σύστημα. Θα πρέπει να διασφαλίζει ότι θα ικανοποιούνται οι ανάγκες των ενδιαφερόμενων και για να γίνει αυτό θα πρέπει να οριστούν με ακρίβεια οι λειτουργικές και οι μη λειτουργικές απαιτήσεις.

Θα παρουσιαστούν οι απαιτήσεις που πρέπει να ικανοποιεί το Πληροφοριακό Σύστημα, οι βασικές λειτουργίες που πρέπει να επιτελεί, οι πληροφορίες που πρέπει να αποθηκεύει και οι κανόνες που επιβάλλονται από την λειτουργία του συστήματος.

%Το παρόν πληροφοριακό σύστημα θα χρησιμοποιηθεί από τους εργαζόμενους της εταιρείας για την κάλυψη των εσωτερικών αναγκών της λειτουργίας της εταιρείας (όπως η διαχείριση των αιθουσών, κ.λ.π.) όσο και από τους πελάτες της εταιρείας για την κράτηση θέσεων μέσω του διαδικτύου. 

\section{Ανάλυση Απαιτήσεων Πληροφοριακού Συστήματος}

Η ανάλυση απαιτήσεων περιλαμβάνει τις εργασίες για τον καθορισμό των αναγκών ή των προϋποθέσεων που χρειάζονται για την ολοκλήρωση ενός προϊόντος (στην συγκεκριμένη περίπτωση του πληροφοριακού συστήματος). Στην ανάλυση απαιτήσεων λαμβάνονται υπόψιν οι ενδεχόμενες αντικρουόμενες απαιτήσεις των διαφόρων μερών ενώ ταυτόχρονα αναλύονται και τεκμηριώνονται οι τυχόν απαιτήσεις του προϊόντος \cite{wiki:requirement_analysis}. Για να είναι επιτυχές ένα πληροφοριακό σύστημα θα πρέπει να είναι προσαρμοσμένο στις ανάγκες, απαιτήσεις, αλλά και προσδοκίες του τελικού χρήστη. Αυτό σημαίνει ότι το ζητούμενο είναι, τί πραγματικά επιθυμεί ο χρήστης, τί ακριβώς περιμένει από το σύστημα και πόσο φιλικό είναι αυτό σε αυτόν και κατά πόσο ικανοποιεί τους σκοπούς για τους οποίους υλοποιήθηκε.  

Οι απαιτήσεις λογισμικού περιλαμβάνουν 3 διαφορετικά επίπεδα \cite{triadis}:

\begin{itemize}
\item Επιχειρηματικές απαιτήσεις
\item Απαιτήσεις χρηστών
\item Λειτουργικές απαιτήσεις
\end{itemize}

Οι επιχειρηματικές απαιτήσεις αντιπροσωπεύουν τους υψηλού επιπέδου στόχους του οργανισμού ή των πελατών που ζητούν το σύστημα. Ορίζουν τον σκοπό και το πεδίο εφαρμογής του νέου συστήματος λογισμικού και περιγράφουν γιατί ο οργανισμός θέλει να εφαρμόσει το σύστημα.  \cite{triadis}.

Οι απαιτήσεις των χρηστών περιγράφουν τους στόχους των χρηστών ή τα καθήκοντα που θα έχουν οι χρήστες στο προϊόν. Οι ανάγκες των χρηστών περιγράφουν τί θα κάνουν οι χρήστες μέσα στο σύστημα. Θα πρέπει να ευθυγραμμίζονται με τις επιχειρηματικές απαιτήσεις. \cite{triadis}

Τέλος οι λειτουργικές απαιτήσεις καθορίζουν την λειτουργικότητα του λογισμικού που πρέπει να φτιάξουν τα μέλη της ομάδας ανάπτυξης έτσι ώστε το προϊόν να επιτρέπει στους χρήστες να εκπληρώνουν τα καθήκοντα τους καλύπτοντας έτσι τις επιχειρησιακές απαιτήσεις \cite{triadis}.

Η ανάλυση απαιτήσεων συντελεί στην καλή οργάνωση και εκτέλεση του έργου, που με τη σειρά τους εξασφαλίζουν τη λειτουργικότητά του για όλες τις εμπλεκόμενες πλευρές. Στο τέλος, τα οφέλη αυτά έχουν άμεσο αντίκρισμα στη μείωση του κόστους, τόσο για την επιχείρηση που υλοποιεί το έργο όσο και για τον πελάτη που θα το χρησιμοποιήσει \cite{kepa:requirement_analysis}.

Οι λειτουργικές απαιτήσεις μαζί με τα χαρακτηριστικά ποιότητας και άλλες μη λειτουργικές απαιτήσεις δημιουργούν την προδιαγραφή των απαιτήσεων λογισμικού \cite{triadis}. 

Η παρούσα αναφορά περιγράφει την βασική λειτουργικότητα και τις σχεδιαστικές αποφάσεις που αφορούν την υλοποίηση του Π.Σ. της εταιρείας "ΚΙΝΗΜΑΤΟΓΡΑΦΟΣ Α.Ε." .

\subsection{Αρχιτεκτονική}

Οι γενικές αρχές, σε λειτουργικό και τεχνολογικό επίπεδο, που θα διέπουν το Π.Σ. που θα αναπτυχθεί είναι:

\begin{enumerate}
\item Συστήματα "ανοικτής" αρχιτεκτονικής (αγγλ. \en{open architecture}). Είναι δηλαδή υποχρεωτική η χρήση ανοικτών προτύπων που θα διασφαλίζουν ανεξαρτησία από συγκεκριμένο προμηθευτή και:

  \begin{itemize}
    \item ομαλή συνεργία και λειτουργία μεταξύ των επιμέρους Υποσυστημάτων του πληρφοριακού συστήματος,
    \item δικτυακή συνεργασία μεταξύ εφαρμογών ή/και συστημάτων τα οποία βρίσκονται σε διαφορετικά υπολογιστικά συστήματα,
    \item επεκτασιμότητα των υποσυστημάτων, χωρίς αλλαγές στη δομή και αρχιτεκτονική τους, για την αντιμετώπιση των μεταβαλλόμενων/αυξανόμενων αναγκών
    \item εύκολη επέμβαση στη λειτουργικότητα των υποσυστημάτων (συντηρισιμότητα - \en{maintainability})
    \item ύψιστη διασφάλιση των δεδομένων.
  \end{itemize}

\item Αρθρωτή αρχιτεκτονική του συστήματος, ώστε να επιτρέπονται μελλοντικές επεκτάσεις και αντικαταστάσεις, ενσωματώσεις, αναβαθμίσεις ή αλλαγές διακριτών τμημάτων λογισμικού ή εξοπλισμού.

\item Εξασφάλιση πλήρους λειτουργικότητας μέσω του εσωτερικού δικτύου (αγγλ. \en{intranet}) και του διαδικτύου (αγγλ. \en{internet}) όπου αυτό απαιτείται.

\item Χρήση γραφικού περιβάλλοντος λειτουργίας (αγγλ. \en{GUI}) του χρήστη για την αποδοτική χρήση του Π.Σ. και την ευκολία εκμάθησης τους.

\item Ενσωμάτωση στο Π.Σ. άμεσης υποστήριξης βοήθειας (αγγλ. \en{online help}) και οδηγιών στην ελληνική γλώσσα, προς τους χρήστες ανά διαδικασία ή/και οθόνη.

\item Μηνύματα λαθών (αγγλ. \en{error messages}) στην ελληνική γλώσσα και ειδοποίηση των χρηστών με όρους οικείους προς αυτούς.

\item Tήρηση από το Π.Σ. στοιχείων auditing για ιχνηλάτηση
ενεργειών χρηστών.

\item Διασφάλιση της πληρότητας, ακεραιότητας, εμπιστευτικότητας και ασφάλειας των δεδομένων των Υποσυστημάτων κατά τη χρήση και τη δικτυακή διακίνησή τους.

\item Τεκμηρίωση του Π.Σ. μέσω της αναλυτικής περιγραφής της βάσης δεδομένων. Σύνταξη τεχνικών εγχειριδίων του συστήματος και των εργαλείων διαχείρισης (αγγλ. \en{system manuals}), καθώς και λεπτομερή εγχειρίδια λειτουργίας του συστήματος (αγγλ. \en{operation manuals} ) και υποστήριξης των χρηστών (αγγλ. \en{user manuals}).

\end{enumerate}

\subsection{Λειτουργικές απαιτήσεις}

Οι λειτουργικές απαιτήσεις είναι οι κύριες δυνατότητες του συστήματος. Αναπαριστούν το "τί" θα κάνει το σύστημα που θα αναπτυχθεί, χωρίς να αναφέρονται στον τρόπο με τον οποίο ("πως") το σύστημα θα το κάνει \cite{triadis}.

Βασική απαίτηση του συστήματος είναι οργάνωση των προβολών των θερινών κινηματογράφων και η αποθήκευση των κρατήσεων των θέσεων ...

Το πληροφορικό σύστημα θα έχεις ως χρηστες τους υπαλλήλου και τους πελάτες

\subsubsection{Λειτουργικότητα}
Η λειτουργικότητα ενός συστήματος μετράται από το πόσο καλά ικανοποιεί τις λειτουργικές απαιτήσεις των ενδιαφερόμενων. Το υποσύστημα για την κράτηση των θέσεων υλοποιεί τη απαιτούμενη μηχανογράφηση για την κράτηση των θέσεων. Βασική απαίτηση από το υποσύστημα είναι η αποθήκευση των απαιτούμενων πληροφοριών για την κράτηση των θέσεων. Από τις πληροφορίες αυτές θα πρέπει να προκύπτουν:

\begin{itemize}
\item Το σύστημα θα πρέπει να επιτρέπει στους χρήστες να ενημερώνουν τις προσωπικές τους πληροφορίες.
\item Θα πρέπει να επιτρέπει την επικοινωνία μεταξύ του πελάτη και των υπαλλήλων της εταιρείας. 

\item αν μία προβολή έχει διαθέσιμες θέσεις
\item πόσοι προκράτησαν τις θέσεις τους για μία ταινία
\item το ιστορικό κρατήσεων ενός πελάτη 
\item ΝΑ ΣΚΕΦΤΩ ΚΑΙ ΑΛΛΛΑ !!!!
\end{itemize}

Ακόμα το υποσύστημα θα πρέπει να επιτρέπει την επισκόπηση των στοιχείων 






Ακόμα το υποσύστημα για την κράτηση των θέσεων θα είναι υπεύθυνο για την πραγματοποίηση ενός ελέγχου για την αποφυγή διπλοκρατήσεων. Συγκεκριμένα πριν την κράτηση μίας θέσεις, θα ελέγχει ότι αυτή η θέση δεν έχει ήδη κρατηθεί.

Όλες οι απαιτούμενες λειτουργικότητες από το υποσύστημα για την κράτηση των θέσεων φαίνονται στο UML διάγραμμα χρήσης (αγγλ. \en{use case diagram}. Οι διαδικασίες (να σκεφτώ ποιες διαδικασίες) παρουσιάζονται αναλυτικότερα στο UML διάγραμμα δραστηριότητας (αγγλ. \en{activity diagram}).

\subsubsection{Περιορισμοί σχεδιασμού}

Οι σχεδιαστές πρέπει να δημιουργήσουν ένα σύστημα το οποίο θα προσαρμόζεται σε κάποιος περιορισμούς. Αν αυτοί οι περιορισμοί δεν συνυπολογιστούν τότε το σύστημα δεν θα λειτουργεί σωστά και θα καταλήξει σε αποτυχία. 


\subsection{Μη λειτουργικές απαιτήσεις}

Οι μη λειτουργικές απαιτήσεις είναι οι περιορισμοί που τίθενται στις λειτουργικές απαιτήσεις, ή στις απαιτήσεις ποιότητας. Αυτές περιλαμβάνουν πληθώρα ιδιοτήτων συμπεριλαμβάνοντας την επίδοση, τους περιορισμούς πολιτικής, την ασφάλεια, την προστασία προσωπικών δεδομένων, την αξιοπιστία. Καθορίζονται γενικά ως ένα βαθμό μετά την μοντελοποίηση των επιχειρηματικών διαδικασιών. Η μοντελοποίηση των μη λειτουργικών χαρακτηριστικών της επιχείρησης θεωρείται ως ένα δύσκολο πρόβλημα, καθώς η μοντελοποίηση επικεντρώνεται στην λειτουργική συμπεριφορά \cite{triadis}.

\subsubsection{Επίδοση}
Η επίδοση έχει να κάνει με περιορισμούς της ταχύτητας που θα πρέπει να εκτελούνται οι διεργασίες, την ποσότητα των δεδομένων που θα αποθηκεύονται και τους χρόνους απόκρισης του συστήματος \cite{triadis}. Παρακάτω υπάρχουν κάποιοι τέτοιοι περιορισμοί:

\begin{description}
\item [Απόκριση:] Οι λειτουργίες του εσωτερικού δικτυακού κόμβου – INTRANET
πρέπει να έχουν χρόνο απόκρισης εντός ολίγων δευτερολέπτων

\item [Εκτέλεση απλών ερωτημάτων:] Το 90\% των συναλλαγών θα πρέπει να ολοκληρώνεται σε χρόνο μικρότερο των 2 sec

\item [Εκτέλεση σύνθετων ερωτημάτων:] To 90\% των συναλλαγών θα πρέπει να ολοκληρώνεται σε χρόνο μικρότερο των 4 sec
\end{description}

\subsubsection{Χρηστικότητα και ανθρώπινοι παράγοντες}

Στην συνέχεια παρουσιάζονται οι παράγοντες που έχουν να κάνουν με την εκπαίδευση των χρηστών και το πόσο εύκολα μπορεί να χρησιμοποιηθεί το σύστημα ως συνάρτηση αυτής της εκπαίδευσης \cite{triadis}

\subsubsection{Ασφάλεια και προστασία προσωπικών πληροφοριών}

Η ασφάλεια είναι ένας κρίσιμος παράγοντας για όλες τις εφαρμογές.

Κατά το σχεδιασμό του Έργου, ο Ανάδοχος θα πρέπει να λάβει ειδική μέριμνα και να δρομολογήσει τις κατάλληλες δράσεις για:
\begin{itemize}
\item την Ασφάλεια των Πληροφοριακών Συστημάτων, Εφαρμογών, Μέσων και Υποδομών
\item την προστασία της ακεραιότητας και της διαθεσιμότητας των πληροφοριών
\item την προστασία των προς επεξεργασία και αποθηκευμένων προσωπικών δεδομένων αναζητώντας και εντοπίζοντας με μεθοδικό τρόπο τα τεχνικά μέτρα και τις οργανωτικο-διοικητικές διαδικασίες
\end{itemize}

\subsubsection{Αξιοπιστία και Διαθεσιμότητα}


\subsubsection{Συντηρησιμότητα}

\subsubsection{Χρηστικότητα - Προσβασιμότητα}

Το σχεδιασμένο σύστημα θα πρέπει να διακρίνεται από υψηλό επίπεδο χρηστικότητας στην οργάνωση και παρουσίαση των ψηφιακών υπηρεσιών που θα παρέχει. Θα πρέπει να λαμβάνονται υπόψιν οι διαφορετικές ομάδες χρηστών κι επομένως τους διαφορετικούς τρόπους εκπλήρωσης της παρεχόμενης λειτουργικότητας χωρίς να μειώνεται η χρηστικότητα των εφαρμογών. Ο σχεδιασμός των εφαρμογών με βασική αρχή την επίτευξη υψηλής χρηστικότητας
και εργονομίας είναι κρίσιμος παράγοντας επιτυχίας για το παρόν έργο.

Οι κυριότερες αρχές προς την κατεύθυνση της χρηστικότητας περιλαμβάνουν:
\begin{description}
\item[Συμβατότητα:] Οι web-εφαρμογές που θα υλοποιηθούν θα πρέπει να είναι
προσβάσιμες με τρεις (3) τουλάχιστον, από τους πιο διαδεδομένους
φυλλομετρητές (web browsers).

\item[Συνέπεια:] Οι εφαρμογές θα πρέπει να έχουν ομοιόμορφη εμφάνιση (κατά το
δυνατόν) και να τηρείται συνέπεια στη χρήση των λεκτικών και των συμβόλων.
Αντίστοιχη συνέπεια πρέπει να επιδεικνύουν οι οποιεσδήποτε γραφικές
απεικονίσεις και οι τοποθετήσεις αντικειμένων. Στο επίπεδο των εφαρμογών
και διαδραστικών λειτουργιών, παρόμοιες λεκτικές και λειτουργικές
απεικονίσεις πρέπει να αντιστοιχούν σε ανάλογα αποτελέσματα.

\item [Αξιοπιστία:] Ο χρήστης πρέπει να έχει σαφείς διαβεβαιώσεις δια μέσου της εμφάνισης και συμπεριφοράς του συστήματος ότι οι συναλλαγές του θα διεκπαραιώνονται με ασφάλεια.

\item[Διαφάνεια:] Ο χρήστης θα πρέπει να  ̈συναλλάσσεται ̈ με το Σύστημα χωρίς να αντιλαμβάνεται τεχνικές λεπτομέρειες ή εσωτερικές διεργασίες διεκπεραίωσης των συναλλαγών.
\end{description}

\subsection{Μοντέλο διαδικασία υλοποίησης και ανάπτυξης του λογισμικού}

Να γράψω για την Rational Unified Process που θα γράψουμε και στην αλλη εργασία.

\section{Μοντέλο Οντοτήτων-Συσχετίσεων της Βάσεις Δεδομένων του Πληροφοριακού Συστήματος}

Στην ενότητα αυτή δίνεται η περιγραφή του μοντέλου Οντοτήτων-Συσχετίσεων (αγγλ. \en{Entity-Relationship Diagram} από το οποίο υλοποιείται το Σχεσιακό Διάγραμμα της Βάσης Δεδομένων του Πληροφοριακού Συστήματος.

\subsection{Περιγραφή Οντοτήτων}
\label{entity}

Οι βασικές οντότητες είναι οι παρακάτω:

\begin{description}
\item test
\end{description}

\subsection{Περιγραφή Σχέσεων}
\label{relationship}

Οι βασικές σχέσεις είναι οι παρακάτω:

\subsection{ER διάγραμμα του Πληροφοριακού Συστήματος}

Οι ενότητες \ref{entity} και \ref{relationship} απεικονίζονται στο σχήμα \ref{}.



\section{Μελέτη Σκοπιμότητας}

Συνοπτικώς αυτό το κεφάλαιο περιλαμβάνει 

\subsection{Περιγραφή του προβλήματος και των εναλλακτικών λύσεων}

Η εταιρεία "ΚΙΝΗΜΑΤΟΓΡΑΦΟΣ Α.Ε." επιθυμεί την ανάπτυξη πληροφοριακού συστήματος για την κάλυψη των αναγκών της. Το πληροφοριακό αυτό σύστημα θα πρέπει να καλύπτει τόσο τις ανάγκες της εσωτερικής της λειτουργίας (διαχείριση αιθουσών και προβολών ταινιών, κ.λ.π.), όσο και την πώληση εισιτηρίων μέσω του διαδικτύου. Η "ΚΙΝΗΜΑΤΟΓΡΑΦΟΣ Α.Ε." είναι ένας από τους μεγαλύτερους ομίλους κινηματογραφικής και όχι μόνο ψυχαγωγίας στην Ελλάδα. 

\subsection{Διαγράμμα Βασικών κλάσεων πεδίου Εφαρμογής}

\subsection{Περιπτώσεις Χρήσης και διαγράμματα περιπτώσεων χρήσης}

\section{Ανάλυση και Σχεδίαση}

\subsection{Αναλυτικές περιπτώσεις και διαγράμματα χρήσης}

\subsection{Αναλυτικά διαγράμματα κλάσεων συνοδευμένα από OCL περιορισμούς}

\subsection{Διαγράμματα Επικοινωνίας και Αλληλεπίδρασης}

\subsection{Περιγραφή των Απαιτήσεων}




\phantomsection \label{Βιβλιογραφία}
\addcontentsline{toc}{section}{Βιβλιογραφία}
%\mtcaddchapter[Βιβλιογραφία] % Λόγω του minitoc
\bibliographystyle{plain}
\bibliography{references}

\newpage

\end{document}

